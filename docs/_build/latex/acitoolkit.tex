% Generated by Sphinx.
\def\sphinxdocclass{report}
\documentclass[letterpaper,10pt,english]{sphinxmanual}
\usepackage[utf8]{inputenc}
\DeclareUnicodeCharacter{00A0}{\nobreakspace}
\usepackage{cmap}
\usepackage[T1]{fontenc}
\usepackage{babel}
\usepackage{times}
\usepackage[Bjarne]{fncychap}
\usepackage{longtable}
\usepackage{sphinx}
\usepackage{multirow}


\title{acitoolkit Documentation}
\date{October 24, 2014}
\release{0.1}
\author{Cisco Systems, Inc.}
\newcommand{\sphinxlogo}{}
\renewcommand{\releasename}{Release}
\makeindex

\makeatletter
\def\PYG@reset{\let\PYG@it=\relax \let\PYG@bf=\relax%
    \let\PYG@ul=\relax \let\PYG@tc=\relax%
    \let\PYG@bc=\relax \let\PYG@ff=\relax}
\def\PYG@tok#1{\csname PYG@tok@#1\endcsname}
\def\PYG@toks#1+{\ifx\relax#1\empty\else%
    \PYG@tok{#1}\expandafter\PYG@toks\fi}
\def\PYG@do#1{\PYG@bc{\PYG@tc{\PYG@ul{%
    \PYG@it{\PYG@bf{\PYG@ff{#1}}}}}}}
\def\PYG#1#2{\PYG@reset\PYG@toks#1+\relax+\PYG@do{#2}}

\expandafter\def\csname PYG@tok@gd\endcsname{\def\PYG@tc##1{\textcolor[rgb]{0.63,0.00,0.00}{##1}}}
\expandafter\def\csname PYG@tok@gu\endcsname{\let\PYG@bf=\textbf\def\PYG@tc##1{\textcolor[rgb]{0.50,0.00,0.50}{##1}}}
\expandafter\def\csname PYG@tok@gt\endcsname{\def\PYG@tc##1{\textcolor[rgb]{0.00,0.27,0.87}{##1}}}
\expandafter\def\csname PYG@tok@gs\endcsname{\let\PYG@bf=\textbf}
\expandafter\def\csname PYG@tok@gr\endcsname{\def\PYG@tc##1{\textcolor[rgb]{1.00,0.00,0.00}{##1}}}
\expandafter\def\csname PYG@tok@cm\endcsname{\let\PYG@it=\textit\def\PYG@tc##1{\textcolor[rgb]{0.25,0.50,0.56}{##1}}}
\expandafter\def\csname PYG@tok@vg\endcsname{\def\PYG@tc##1{\textcolor[rgb]{0.73,0.38,0.84}{##1}}}
\expandafter\def\csname PYG@tok@m\endcsname{\def\PYG@tc##1{\textcolor[rgb]{0.13,0.50,0.31}{##1}}}
\expandafter\def\csname PYG@tok@mh\endcsname{\def\PYG@tc##1{\textcolor[rgb]{0.13,0.50,0.31}{##1}}}
\expandafter\def\csname PYG@tok@cs\endcsname{\def\PYG@tc##1{\textcolor[rgb]{0.25,0.50,0.56}{##1}}\def\PYG@bc##1{\setlength{\fboxsep}{0pt}\colorbox[rgb]{1.00,0.94,0.94}{\strut ##1}}}
\expandafter\def\csname PYG@tok@ge\endcsname{\let\PYG@it=\textit}
\expandafter\def\csname PYG@tok@vc\endcsname{\def\PYG@tc##1{\textcolor[rgb]{0.73,0.38,0.84}{##1}}}
\expandafter\def\csname PYG@tok@il\endcsname{\def\PYG@tc##1{\textcolor[rgb]{0.13,0.50,0.31}{##1}}}
\expandafter\def\csname PYG@tok@go\endcsname{\def\PYG@tc##1{\textcolor[rgb]{0.20,0.20,0.20}{##1}}}
\expandafter\def\csname PYG@tok@cp\endcsname{\def\PYG@tc##1{\textcolor[rgb]{0.00,0.44,0.13}{##1}}}
\expandafter\def\csname PYG@tok@gi\endcsname{\def\PYG@tc##1{\textcolor[rgb]{0.00,0.63,0.00}{##1}}}
\expandafter\def\csname PYG@tok@gh\endcsname{\let\PYG@bf=\textbf\def\PYG@tc##1{\textcolor[rgb]{0.00,0.00,0.50}{##1}}}
\expandafter\def\csname PYG@tok@ni\endcsname{\let\PYG@bf=\textbf\def\PYG@tc##1{\textcolor[rgb]{0.84,0.33,0.22}{##1}}}
\expandafter\def\csname PYG@tok@nl\endcsname{\let\PYG@bf=\textbf\def\PYG@tc##1{\textcolor[rgb]{0.00,0.13,0.44}{##1}}}
\expandafter\def\csname PYG@tok@nn\endcsname{\let\PYG@bf=\textbf\def\PYG@tc##1{\textcolor[rgb]{0.05,0.52,0.71}{##1}}}
\expandafter\def\csname PYG@tok@no\endcsname{\def\PYG@tc##1{\textcolor[rgb]{0.38,0.68,0.84}{##1}}}
\expandafter\def\csname PYG@tok@na\endcsname{\def\PYG@tc##1{\textcolor[rgb]{0.25,0.44,0.63}{##1}}}
\expandafter\def\csname PYG@tok@nb\endcsname{\def\PYG@tc##1{\textcolor[rgb]{0.00,0.44,0.13}{##1}}}
\expandafter\def\csname PYG@tok@nc\endcsname{\let\PYG@bf=\textbf\def\PYG@tc##1{\textcolor[rgb]{0.05,0.52,0.71}{##1}}}
\expandafter\def\csname PYG@tok@nd\endcsname{\let\PYG@bf=\textbf\def\PYG@tc##1{\textcolor[rgb]{0.33,0.33,0.33}{##1}}}
\expandafter\def\csname PYG@tok@ne\endcsname{\def\PYG@tc##1{\textcolor[rgb]{0.00,0.44,0.13}{##1}}}
\expandafter\def\csname PYG@tok@nf\endcsname{\def\PYG@tc##1{\textcolor[rgb]{0.02,0.16,0.49}{##1}}}
\expandafter\def\csname PYG@tok@si\endcsname{\let\PYG@it=\textit\def\PYG@tc##1{\textcolor[rgb]{0.44,0.63,0.82}{##1}}}
\expandafter\def\csname PYG@tok@s2\endcsname{\def\PYG@tc##1{\textcolor[rgb]{0.25,0.44,0.63}{##1}}}
\expandafter\def\csname PYG@tok@vi\endcsname{\def\PYG@tc##1{\textcolor[rgb]{0.73,0.38,0.84}{##1}}}
\expandafter\def\csname PYG@tok@nt\endcsname{\let\PYG@bf=\textbf\def\PYG@tc##1{\textcolor[rgb]{0.02,0.16,0.45}{##1}}}
\expandafter\def\csname PYG@tok@nv\endcsname{\def\PYG@tc##1{\textcolor[rgb]{0.73,0.38,0.84}{##1}}}
\expandafter\def\csname PYG@tok@s1\endcsname{\def\PYG@tc##1{\textcolor[rgb]{0.25,0.44,0.63}{##1}}}
\expandafter\def\csname PYG@tok@gp\endcsname{\let\PYG@bf=\textbf\def\PYG@tc##1{\textcolor[rgb]{0.78,0.36,0.04}{##1}}}
\expandafter\def\csname PYG@tok@sh\endcsname{\def\PYG@tc##1{\textcolor[rgb]{0.25,0.44,0.63}{##1}}}
\expandafter\def\csname PYG@tok@ow\endcsname{\let\PYG@bf=\textbf\def\PYG@tc##1{\textcolor[rgb]{0.00,0.44,0.13}{##1}}}
\expandafter\def\csname PYG@tok@sx\endcsname{\def\PYG@tc##1{\textcolor[rgb]{0.78,0.36,0.04}{##1}}}
\expandafter\def\csname PYG@tok@bp\endcsname{\def\PYG@tc##1{\textcolor[rgb]{0.00,0.44,0.13}{##1}}}
\expandafter\def\csname PYG@tok@c1\endcsname{\let\PYG@it=\textit\def\PYG@tc##1{\textcolor[rgb]{0.25,0.50,0.56}{##1}}}
\expandafter\def\csname PYG@tok@kc\endcsname{\let\PYG@bf=\textbf\def\PYG@tc##1{\textcolor[rgb]{0.00,0.44,0.13}{##1}}}
\expandafter\def\csname PYG@tok@c\endcsname{\let\PYG@it=\textit\def\PYG@tc##1{\textcolor[rgb]{0.25,0.50,0.56}{##1}}}
\expandafter\def\csname PYG@tok@mf\endcsname{\def\PYG@tc##1{\textcolor[rgb]{0.13,0.50,0.31}{##1}}}
\expandafter\def\csname PYG@tok@err\endcsname{\def\PYG@bc##1{\setlength{\fboxsep}{0pt}\fcolorbox[rgb]{1.00,0.00,0.00}{1,1,1}{\strut ##1}}}
\expandafter\def\csname PYG@tok@kd\endcsname{\let\PYG@bf=\textbf\def\PYG@tc##1{\textcolor[rgb]{0.00,0.44,0.13}{##1}}}
\expandafter\def\csname PYG@tok@ss\endcsname{\def\PYG@tc##1{\textcolor[rgb]{0.32,0.47,0.09}{##1}}}
\expandafter\def\csname PYG@tok@sr\endcsname{\def\PYG@tc##1{\textcolor[rgb]{0.14,0.33,0.53}{##1}}}
\expandafter\def\csname PYG@tok@mo\endcsname{\def\PYG@tc##1{\textcolor[rgb]{0.13,0.50,0.31}{##1}}}
\expandafter\def\csname PYG@tok@mi\endcsname{\def\PYG@tc##1{\textcolor[rgb]{0.13,0.50,0.31}{##1}}}
\expandafter\def\csname PYG@tok@kn\endcsname{\let\PYG@bf=\textbf\def\PYG@tc##1{\textcolor[rgb]{0.00,0.44,0.13}{##1}}}
\expandafter\def\csname PYG@tok@o\endcsname{\def\PYG@tc##1{\textcolor[rgb]{0.40,0.40,0.40}{##1}}}
\expandafter\def\csname PYG@tok@kr\endcsname{\let\PYG@bf=\textbf\def\PYG@tc##1{\textcolor[rgb]{0.00,0.44,0.13}{##1}}}
\expandafter\def\csname PYG@tok@s\endcsname{\def\PYG@tc##1{\textcolor[rgb]{0.25,0.44,0.63}{##1}}}
\expandafter\def\csname PYG@tok@kp\endcsname{\def\PYG@tc##1{\textcolor[rgb]{0.00,0.44,0.13}{##1}}}
\expandafter\def\csname PYG@tok@w\endcsname{\def\PYG@tc##1{\textcolor[rgb]{0.73,0.73,0.73}{##1}}}
\expandafter\def\csname PYG@tok@kt\endcsname{\def\PYG@tc##1{\textcolor[rgb]{0.56,0.13,0.00}{##1}}}
\expandafter\def\csname PYG@tok@sc\endcsname{\def\PYG@tc##1{\textcolor[rgb]{0.25,0.44,0.63}{##1}}}
\expandafter\def\csname PYG@tok@sb\endcsname{\def\PYG@tc##1{\textcolor[rgb]{0.25,0.44,0.63}{##1}}}
\expandafter\def\csname PYG@tok@k\endcsname{\let\PYG@bf=\textbf\def\PYG@tc##1{\textcolor[rgb]{0.00,0.44,0.13}{##1}}}
\expandafter\def\csname PYG@tok@se\endcsname{\let\PYG@bf=\textbf\def\PYG@tc##1{\textcolor[rgb]{0.25,0.44,0.63}{##1}}}
\expandafter\def\csname PYG@tok@sd\endcsname{\let\PYG@it=\textit\def\PYG@tc##1{\textcolor[rgb]{0.25,0.44,0.63}{##1}}}

\def\PYGZbs{\char`\\}
\def\PYGZus{\char`\_}
\def\PYGZob{\char`\{}
\def\PYGZcb{\char`\}}
\def\PYGZca{\char`\^}
\def\PYGZam{\char`\&}
\def\PYGZlt{\char`\<}
\def\PYGZgt{\char`\>}
\def\PYGZsh{\char`\#}
\def\PYGZpc{\char`\%}
\def\PYGZdl{\char`\$}
\def\PYGZhy{\char`\-}
\def\PYGZsq{\char`\'}
\def\PYGZdq{\char`\"}
\def\PYGZti{\char`\~}
% for compatibility with earlier versions
\def\PYGZat{@}
\def\PYGZlb{[}
\def\PYGZrb{]}
\makeatother

\renewcommand\PYGZsq{\textquotesingle}

\begin{document}

\maketitle
\tableofcontents
\phantomsection\label{index::doc}


Contents:


\chapter{acitoolkit}
\label{modules:welcome-to-acitoolkit-s-documentation}\label{modules::doc}\label{modules:acitoolkit}

\section{acibaseobject module}
\label{acibaseobject:module-acibaseobject}\label{acibaseobject::doc}\label{acibaseobject:acibaseobject-module}\index{acibaseobject (module)}
This module implements the Base Class for creating all of the ACI Objects
\index{BaseACIObject (class in acibaseobject)}

\begin{fulllineitems}
\phantomsection\label{acibaseobject:acibaseobject.BaseACIObject}\pysiglinewithargsret{\strong{class }\code{acibaseobject.}\bfcode{BaseACIObject}}{\emph{name}, \emph{parent=None}}{}
Bases: \code{object}

This class defines functionality common to all ACI objects.
Functions may be overwritten by inheriting classes.

Initialize the basic object.  This should be called by the
init routines of inheriting subclasses.
\index{add\_child() (acibaseobject.BaseACIObject method)}

\begin{fulllineitems}
\phantomsection\label{acibaseobject:acibaseobject.BaseACIObject.add_child}\pysiglinewithargsret{\bfcode{add\_child}}{\emph{obj}}{}
Add a child to the children list

\end{fulllineitems}

\index{attach() (acibaseobject.BaseACIObject method)}

\begin{fulllineitems}
\phantomsection\label{acibaseobject:acibaseobject.BaseACIObject.attach}\pysiglinewithargsret{\bfcode{attach}}{\emph{item}}{}
Attach the object to the other object

\end{fulllineitems}

\index{detach() (acibaseobject.BaseACIObject method)}

\begin{fulllineitems}
\phantomsection\label{acibaseobject:acibaseobject.BaseACIObject.detach}\pysiglinewithargsret{\bfcode{detach}}{\emph{item}}{}
Detach the object from the other object

\end{fulllineitems}

\index{find() (acibaseobject.BaseACIObject method)}

\begin{fulllineitems}
\phantomsection\label{acibaseobject:acibaseobject.BaseACIObject.find}\pysiglinewithargsret{\bfcode{find}}{\emph{search\_object}}{}~\begin{description}
\item[{This will check to see if self is a match with search\_object}] \leavevmode
and then call find on all of the children of search.
If there is a match, a list containing self and any matches found
by the children will be returned as a list.

The criteria for a match is that all attributes of self are
compared to all attributes of search\_object.
If search\_object.\textless{}attr\textgreater{} exists and is the same as self.\textless{}attr\textgreater{} or
search\_object.\textless{}attr\textgreater{} is `None', then that attribute matches.
If all such attributes match, then there is a match and self will
be returned in the result.

If there is an attribute of search\_object that does not exist in
self, it will be considered a mismatch.
If there is an attribute of self that does not exist in
search\_object, it will be ignored.

\end{description}

INPUT: search\_object = aci object

RETURNS: list of objects

\end{fulllineitems}

\index{get() (acibaseobject.BaseACIObject class method)}

\begin{fulllineitems}
\phantomsection\label{acibaseobject:acibaseobject.BaseACIObject.get}\pysiglinewithargsret{\strong{classmethod }\bfcode{get}}{\emph{session}, \emph{toolkit\_class}, \emph{apic\_class}, \emph{parent=None}, \emph{tenant=None}}{}
Gets all of a particular class.

\end{fulllineitems}

\index{get\_all\_attached() (acibaseobject.BaseACIObject method)}

\begin{fulllineitems}
\phantomsection\label{acibaseobject:acibaseobject.BaseACIObject.get_all_attached}\pysiglinewithargsret{\bfcode{get\_all\_attached}}{\emph{attached\_class}, \emph{status='attached'}}{}
Get all of the relations of objects beloging to the
specified class

\end{fulllineitems}

\index{get\_children() (acibaseobject.BaseACIObject method)}

\begin{fulllineitems}
\phantomsection\label{acibaseobject:acibaseobject.BaseACIObject.get_children}\pysiglinewithargsret{\bfcode{get\_children}}{}{}
Returns the list of children

\end{fulllineitems}

\index{get\_interfaces() (acibaseobject.BaseACIObject method)}

\begin{fulllineitems}
\phantomsection\label{acibaseobject:acibaseobject.BaseACIObject.get_interfaces}\pysiglinewithargsret{\bfcode{get\_interfaces}}{\emph{status='attached'}}{}
Get all of the interface relations

\end{fulllineitems}

\index{get\_json() (acibaseobject.BaseACIObject method)}

\begin{fulllineitems}
\phantomsection\label{acibaseobject:acibaseobject.BaseACIObject.get_json}\pysiglinewithargsret{\bfcode{get\_json}}{\emph{obj\_class}, \emph{attributes=None}, \emph{children=None}, \emph{get\_children=True}}{}
Get the JSON representation of this class in the actual APIC
Object Model

\end{fulllineitems}

\index{get\_parent() (acibaseobject.BaseACIObject method)}

\begin{fulllineitems}
\phantomsection\label{acibaseobject:acibaseobject.BaseACIObject.get_parent}\pysiglinewithargsret{\bfcode{get\_parent}}{}{}
Returns the parent of this object

\end{fulllineitems}

\index{has\_child() (acibaseobject.BaseACIObject method)}

\begin{fulllineitems}
\phantomsection\label{acibaseobject:acibaseobject.BaseACIObject.has_child}\pysiglinewithargsret{\bfcode{has\_child}}{\emph{obj}}{}
Check for existence of a child in the children list

\end{fulllineitems}

\index{info() (acibaseobject.BaseACIObject method)}

\begin{fulllineitems}
\phantomsection\label{acibaseobject:acibaseobject.BaseACIObject.info}\pysiglinewithargsret{\bfcode{info}}{}{}~\begin{description}
\item[{this will return a formatted string that has a summary of all}] \leavevmode
the info gathered about the node.

\end{description}

INPUT: None

RETURNS: str

\end{fulllineitems}

\index{is\_attached() (acibaseobject.BaseACIObject method)}

\begin{fulllineitems}
\phantomsection\label{acibaseobject:acibaseobject.BaseACIObject.is_attached}\pysiglinewithargsret{\bfcode{is\_attached}}{\emph{item}}{}
Returns True if the item is attached to this object

\end{fulllineitems}

\index{is\_deleted() (acibaseobject.BaseACIObject method)}

\begin{fulllineitems}
\phantomsection\label{acibaseobject:acibaseobject.BaseACIObject.is_deleted}\pysiglinewithargsret{\bfcode{is\_deleted}}{}{}
Check if the object has been deleted.

\end{fulllineitems}

\index{is\_interface() (acibaseobject.BaseACIObject static method)}

\begin{fulllineitems}
\phantomsection\label{acibaseobject:acibaseobject.BaseACIObject.is_interface}\pysiglinewithargsret{\strong{static }\bfcode{is\_interface}}{}{}
Return whether this object is considered an Interface

RETURN: False

\end{fulllineitems}

\index{mark\_as\_deleted() (acibaseobject.BaseACIObject method)}

\begin{fulllineitems}
\phantomsection\label{acibaseobject:acibaseobject.BaseACIObject.mark_as_deleted}\pysiglinewithargsret{\bfcode{mark\_as\_deleted}}{}{}
Mark the object as deleted.  This will cause the JSON status
to be set to deleted

\end{fulllineitems}

\index{populate\_children() (acibaseobject.BaseACIObject method)}

\begin{fulllineitems}
\phantomsection\label{acibaseobject:acibaseobject.BaseACIObject.populate_children}\pysiglinewithargsret{\bfcode{populate\_children}}{\emph{deep=False}}{}
Populates all of the children and then calls populate\_children
of those children if deep is True.  This method should be
overridden by any object that does have children

\end{fulllineitems}

\index{remove\_child() (acibaseobject.BaseACIObject method)}

\begin{fulllineitems}
\phantomsection\label{acibaseobject:acibaseobject.BaseACIObject.remove_child}\pysiglinewithargsret{\bfcode{remove\_child}}{\emph{obj}}{}
Remove a child from the children list

\end{fulllineitems}


\end{fulllineitems}

\index{BaseRelation (class in acibaseobject)}

\begin{fulllineitems}
\phantomsection\label{acibaseobject:acibaseobject.BaseRelation}\pysiglinewithargsret{\strong{class }\code{acibaseobject.}\bfcode{BaseRelation}}{\emph{item}, \emph{status}, \emph{relation\_type=None}}{}
Bases: \code{object}

Class for all basic relations.
A relation consists of the following elements:
item:    The object to which the relationship applies
status:  The status of the relationship
\begin{quote}

Valid values are `attached' and `detached'
\end{quote}
\begin{description}
\item[{relation\_type:   Additional information to distinguish the relationship.}] \leavevmode
Used in cases where more than 1 type of relation exists

\end{description}
\index{is\_attached() (acibaseobject.BaseRelation method)}

\begin{fulllineitems}
\phantomsection\label{acibaseobject:acibaseobject.BaseRelation.is_attached}\pysiglinewithargsret{\bfcode{is\_attached}}{}{}
Returns whether the relation is attached.
If a relation is detached, it is to be deleted from the APIC

\end{fulllineitems}

\index{is\_detached() (acibaseobject.BaseRelation method)}

\begin{fulllineitems}
\phantomsection\label{acibaseobject:acibaseobject.BaseRelation.is_detached}\pysiglinewithargsret{\bfcode{is\_detached}}{}{}
Returns whether the relation is detached.
If a relation is detached, it is to be deleted from the APIC

\end{fulllineitems}

\index{set\_as\_detached() (acibaseobject.BaseRelation method)}

\begin{fulllineitems}
\phantomsection\label{acibaseobject:acibaseobject.BaseRelation.set_as_detached}\pysiglinewithargsret{\bfcode{set\_as\_detached}}{}{}
Set the relation as detached

\end{fulllineitems}


\end{fulllineitems}



\section{aciphysobject module}
\label{aciphysobject:aciphysobject-module}\label{aciphysobject::doc}\label{aciphysobject:module-aciphysobject}\index{aciphysobject (module)}
ACI Toolkit module for physical objects
\index{BaseACIPhysModule (class in aciphysobject)}

\begin{fulllineitems}
\phantomsection\label{aciphysobject:aciphysobject.BaseACIPhysModule}\pysiglinewithargsret{\strong{class }\code{aciphysobject.}\bfcode{BaseACIPhysModule}}{\emph{pod}, \emph{node}, \emph{slot}, \emph{parent=None}}{}
Bases: {\hyperref[aciphysobject:aciphysobject.BaseACIPhysObject]{\code{aciphysobject.BaseACIPhysObject}}}

BaseACIPhysModule: base class for modules

Initialize the basic object.  This should be called by the
init routines of inheriting subclasses.
\index{get\_obj() (aciphysobject.BaseACIPhysModule class method)}

\begin{fulllineitems}
\phantomsection\label{aciphysobject:aciphysobject.BaseACIPhysModule.get_obj}\pysiglinewithargsret{\strong{classmethod }\bfcode{get\_obj}}{\emph{session}, \emph{apic\_class}, \emph{parent}}{}
Gets all of the Nodes from the APIC.  This is called by the
module specific get() methods.  The parameters passed include the
APIC object class, apic\_class, so that this will work for different kinds of modules.

INPUT: session, apic\_class = object class name in APIC, parent
OUTPUT: list of module objects derived from the specified apic\_class

\end{fulllineitems}

\index{get\_serial() (aciphysobject.BaseACIPhysModule method)}

\begin{fulllineitems}
\phantomsection\label{aciphysobject:aciphysobject.BaseACIPhysModule.get_serial}\pysiglinewithargsret{\bfcode{get\_serial}}{}{}
returns the serial number

\end{fulllineitems}

\index{get\_slot() (aciphysobject.BaseACIPhysModule method)}

\begin{fulllineitems}
\phantomsection\label{aciphysobject:aciphysobject.BaseACIPhysModule.get_slot}\pysiglinewithargsret{\bfcode{get\_slot}}{}{}
Gets slot id

\end{fulllineitems}


\end{fulllineitems}

\index{BaseACIPhysObject (class in aciphysobject)}

\begin{fulllineitems}
\phantomsection\label{aciphysobject:aciphysobject.BaseACIPhysObject}\pysiglinewithargsret{\strong{class }\code{aciphysobject.}\bfcode{BaseACIPhysObject}}{\emph{name}, \emph{parent=None}}{}
Bases: {\hyperref[acibaseobject:acibaseobject.BaseACIObject]{\code{acibaseobject.BaseACIObject}}}

Base class for physical objects

Initialize the basic object.  This should be called by the
init routines of inheriting subclasses.
\index{add\_child() (aciphysobject.BaseACIPhysObject method)}

\begin{fulllineitems}
\phantomsection\label{aciphysobject:aciphysobject.BaseACIPhysObject.add_child}\pysiglinewithargsret{\bfcode{add\_child}}{\emph{child\_obj}}{}
Add a child to the children list. All children must be unique so it will
first delete the child if it already exists.

INPUT: child\_obj

RETURN: None

\end{fulllineitems}

\index{exists() (aciphysobject.BaseACIPhysObject class method)}

\begin{fulllineitems}
\phantomsection\label{aciphysobject:aciphysobject.BaseACIPhysObject.exists}\pysiglinewithargsret{\strong{classmethod }\bfcode{exists}}{\emph{session}, \emph{phys\_obj}}{}
Check if an apic phys\_obj exists on the APIC.
Returns True if the phys\_obj does exist.

INPUT: session, phys\_obj

RETURNS: boolean

\end{fulllineitems}

\index{get\_children() (aciphysobject.BaseACIPhysObject method)}

\begin{fulllineitems}
\phantomsection\label{aciphysobject:aciphysobject.BaseACIPhysObject.get_children}\pysiglinewithargsret{\bfcode{get\_children}}{\emph{childType=None}}{}
Returns the list of children.  If childType is provided, then
it will return all of the children of the matching type.

INPUT: optional childType

RETURNS: list of children

\end{fulllineitems}

\index{get\_json() (aciphysobject.BaseACIPhysObject method)}

\begin{fulllineitems}
\phantomsection\label{aciphysobject:aciphysobject.BaseACIPhysObject.get_json}\pysiglinewithargsret{\bfcode{get\_json}}{}{}
Returns json representation of the object

INPUT:
RETURNS: Nothing - physical objects are not modifiable

\end{fulllineitems}

\index{get\_name() (aciphysobject.BaseACIPhysObject method)}

\begin{fulllineitems}
\phantomsection\label{aciphysobject:aciphysobject.BaseACIPhysObject.get_name}\pysiglinewithargsret{\bfcode{get\_name}}{}{}
Gets name

\end{fulllineitems}

\index{get\_node() (aciphysobject.BaseACIPhysObject method)}

\begin{fulllineitems}
\phantomsection\label{aciphysobject:aciphysobject.BaseACIPhysObject.get_node}\pysiglinewithargsret{\bfcode{get\_node}}{}{}
Gets node id

\end{fulllineitems}

\index{get\_pod() (aciphysobject.BaseACIPhysObject method)}

\begin{fulllineitems}
\phantomsection\label{aciphysobject:aciphysobject.BaseACIPhysObject.get_pod}\pysiglinewithargsret{\bfcode{get\_pod}}{}{}
Gets pod id

\end{fulllineitems}

\index{get\_serial() (aciphysobject.BaseACIPhysObject method)}

\begin{fulllineitems}
\phantomsection\label{aciphysobject:aciphysobject.BaseACIPhysObject.get_serial}\pysiglinewithargsret{\bfcode{get\_serial}}{}{}
\end{fulllineitems}

\index{get\_type() (aciphysobject.BaseACIPhysObject method)}

\begin{fulllineitems}
\phantomsection\label{aciphysobject:aciphysobject.BaseACIPhysObject.get_type}\pysiglinewithargsret{\bfcode{get\_type}}{}{}
Gets physical object type

\end{fulllineitems}

\index{get\_url() (aciphysobject.BaseACIPhysObject method)}

\begin{fulllineitems}
\phantomsection\label{aciphysobject:aciphysobject.BaseACIPhysObject.get_url}\pysiglinewithargsret{\bfcode{get\_url}}{\emph{fmt='json'}}{}
Get the URL used to push the configuration to the APIC
if no fmt parameter is specified, the format will be `json'
otherwise it will return `/api/mo/uni.' with the fmt string appended.

INPUT: optional fmt string
RETURNS: Nothing - physical objects are not modifiable

\end{fulllineitems}


\end{fulllineitems}

\index{Fantray (class in aciphysobject)}

\begin{fulllineitems}
\phantomsection\label{aciphysobject:aciphysobject.Fantray}\pysiglinewithargsret{\strong{class }\code{aciphysobject.}\bfcode{Fantray}}{\emph{pod}, \emph{node}, \emph{slot}, \emph{parent=None}}{}
Bases: {\hyperref[aciphysobject:aciphysobject.BaseACIPhysModule]{\code{aciphysobject.BaseACIPhysModule}}}

class for the fan tray of a node

Initialize the basic object.  It will create the name of the fan tray and set the type
before calling the base class \_\_init\_\_ method
\index{get() (aciphysobject.Fantray class method)}

\begin{fulllineitems}
\phantomsection\label{aciphysobject:aciphysobject.Fantray.get}\pysiglinewithargsret{\strong{classmethod }\bfcode{get}}{\emph{session}, \emph{parent=None}}{}
Gets all of the fantrays from the APIC.  If parent is specified, it will only get fantrays that are
children of the the parent.  The fantrays will also be added as children to the parent Node.

The fantray object is derived mostly from the APIC `eqptFt' class.

INPUT: session, parent of class Node

OUTPUT: list of fantrays

\end{fulllineitems}

\index{populate\_children() (aciphysobject.Fantray method)}

\begin{fulllineitems}
\phantomsection\label{aciphysobject:aciphysobject.Fantray.populate_children}\pysiglinewithargsret{\bfcode{populate\_children}}{\emph{deep=False}}{}
\end{fulllineitems}


\end{fulllineitems}

\index{Linecard (class in aciphysobject)}

\begin{fulllineitems}
\phantomsection\label{aciphysobject:aciphysobject.Linecard}\pysiglinewithargsret{\strong{class }\code{aciphysobject.}\bfcode{Linecard}}{\emph{arg0=None}, \emph{arg1=None}, \emph{slot=None}, \emph{parent=None}}{}
Bases: {\hyperref[aciphysobject:aciphysobject.BaseACIPhysModule]{\code{aciphysobject.BaseACIPhysModule}}}

class for a linecard of a switch

Initialize the basic object.  It will create the name of the linecard and set the type
before calling the base class \_\_init\_\_ method.  If arg1 is an instance of a Node, then pod,
and node are derived from the Node and the slot\_id is from arg0.  If arg1 is not a Node, then arg0
is the pod, arg1 is the node id, and slot is the slot\_id

INPUT: arg0=str, arg0={[}str,Node{]}, {[}slot=str{]}, {[}parent=Node{]}

RETURNS: None
\index{get() (aciphysobject.Linecard class method)}

\begin{fulllineitems}
\phantomsection\label{aciphysobject:aciphysobject.Linecard.get}\pysiglinewithargsret{\strong{classmethod }\bfcode{get}}{\emph{session}, \emph{parent=None}}{}
Gets all of the linecards from the APIC.  If parent is specified, it will only get linecards that are
children of the the parent.  The linecards will also be added as children to the parent Node.

The lincard object is derived mostly from the APIC `eqptLC' class.

INPUT: session, parent of class Node

OUTPUT: list of linecards

\end{fulllineitems}

\index{populate\_children() (aciphysobject.Linecard method)}

\begin{fulllineitems}
\phantomsection\label{aciphysobject:aciphysobject.Linecard.populate_children}\pysiglinewithargsret{\bfcode{populate\_children}}{\emph{deep=False}}{}
populates all of the children of the linecard.  Children are the interfaces.
If deep is set to true, it will also try to populate the children of the children.

INPUT: {[}boolean{]}

RETURNS: None

\end{fulllineitems}


\end{fulllineitems}

\index{Link (class in aciphysobject)}

\begin{fulllineitems}
\phantomsection\label{aciphysobject:aciphysobject.Link}\pysiglinewithargsret{\strong{class }\code{aciphysobject.}\bfcode{Link}}{\emph{pod}, \emph{link}, \emph{node1}, \emph{slot1}, \emph{port1}, \emph{node2}, \emph{slot2}, \emph{port2}, \emph{parent=None}}{}
Bases: {\hyperref[aciphysobject:aciphysobject.BaseACIPhysObject]{\code{aciphysobject.BaseACIPhysObject}}}

Link class, equivalent to the fabricLink object in APIC
\index{get() (aciphysobject.Link static method)}

\begin{fulllineitems}
\phantomsection\label{aciphysobject:aciphysobject.Link.get}\pysiglinewithargsret{\strong{static }\bfcode{get}}{\emph{session}, \emph{parent=None}}{}
Gets all of the Links from the APIC.  If the parent pod is specified,
only links of that pod will be retrieved.

\end{fulllineitems}

\index{get\_linkstatus() (aciphysobject.Link method)}

\begin{fulllineitems}
\phantomsection\label{aciphysobject:aciphysobject.Link.get_linkstatus}\pysiglinewithargsret{\bfcode{get\_linkstatus}}{}{}
\end{fulllineitems}

\index{get\_node1() (aciphysobject.Link method)}

\begin{fulllineitems}
\phantomsection\label{aciphysobject:aciphysobject.Link.get_node1}\pysiglinewithargsret{\bfcode{get\_node1}}{}{}
Returns the Node object that corresponds to the first node of the link.  The Node must be a child of
the Pod that this link is a member of, i.e. it must already have been read from the APIC.  This can
most easily be done by populating the entire physical heirarchy from the Pod down.

INPUT: None

OUTPUT: Node

\end{fulllineitems}

\index{get\_node2() (aciphysobject.Link method)}

\begin{fulllineitems}
\phantomsection\label{aciphysobject:aciphysobject.Link.get_node2}\pysiglinewithargsret{\bfcode{get\_node2}}{}{}
Returns the Node object that corresponds to the second node of the link.  The Node must be a child of
the Pod that this link is a member of, i.e. it must already have been read from the APIC.  This can
most easily be done by populating the entire physical heirarchy from the Pod down.

INPUT: None

OUTPUT: Node

\end{fulllineitems}

\index{get\_port1() (aciphysobject.Link method)}

\begin{fulllineitems}
\phantomsection\label{aciphysobject:aciphysobject.Link.get_port1}\pysiglinewithargsret{\bfcode{get\_port1}}{}{}
Returns the Linecard object that corresponds to the first port of the link.  The port must be a child of
the Linecard in the Node in the Pod that this link is a member of, i.e. it must already have been read from the APIC.  This can
most easily be done by populating the entire physical heirarchy from the Pod down.

INPUT: None

OUTPUT: Interface

\end{fulllineitems}

\index{get\_port2() (aciphysobject.Link method)}

\begin{fulllineitems}
\phantomsection\label{aciphysobject:aciphysobject.Link.get_port2}\pysiglinewithargsret{\bfcode{get\_port2}}{}{}
Returns the Linecard object that corresponds to the second port of the link.  The port must be a child of
the Linecard in the Node in the Pod that this link is a member of, i.e. it must already have been read from the APIC.  This can
most easily be done by populating the entire physical heirarchy from the Pod down.

INPUT: None

OUTPUT: Interface

\end{fulllineitems}

\index{get\_slot1() (aciphysobject.Link method)}

\begin{fulllineitems}
\phantomsection\label{aciphysobject:aciphysobject.Link.get_slot1}\pysiglinewithargsret{\bfcode{get\_slot1}}{}{}
Returns the Linecard object that corresponds to the first slot of the link.  The Linecard must be a child of
the Node in the Pod that this link is a member of, i.e. it must already have been read from the APIC.  This can
most easily be done by populating the entire physical heirarchy from the Pod down.

INPUT: None

OUTPUT: Node

\end{fulllineitems}

\index{get\_slot2() (aciphysobject.Link method)}

\begin{fulllineitems}
\phantomsection\label{aciphysobject:aciphysobject.Link.get_slot2}\pysiglinewithargsret{\bfcode{get\_slot2}}{}{}
Returns the Linecard object that corresponds to the second slot of the link.  The Linecard must be a child of
the Node in the Pod that this link is a member of, i.e. it must already have been read from the APIC.  This can
most easily be done by populating the entire physical heirarchy from the Pod down.

INPUT: None

OUTPUT: Node

\end{fulllineitems}


\end{fulllineitems}

\index{Node (class in aciphysobject)}

\begin{fulllineitems}
\phantomsection\label{aciphysobject:aciphysobject.Node}\pysiglinewithargsret{\strong{class }\code{aciphysobject.}\bfcode{Node}}{\emph{pod=None}, \emph{node=None}, \emph{name=None}, \emph{role=None}, \emph{parent=None}}{}
Bases: {\hyperref[aciphysobject:aciphysobject.BaseACIPhysObject]{\code{aciphysobject.BaseACIPhysObject}}}

Node :  roughly equivalent to eqptNode

Initialize the basic object.
\index{get() (aciphysobject.Node static method)}

\begin{fulllineitems}
\phantomsection\label{aciphysobject:aciphysobject.Node.get}\pysiglinewithargsret{\strong{static }\bfcode{get}}{\emph{session}, \emph{parent=None}}{}
Gets all of the Nodes from the APIC.  If the parent pod is specified,
only nodes of that pod will be retrieved.

\end{fulllineitems}

\index{get\_chassisType() (aciphysobject.Node method)}

\begin{fulllineitems}
\phantomsection\label{aciphysobject:aciphysobject.Node.get_chassisType}\pysiglinewithargsret{\bfcode{get\_chassisType}}{}{}
returns the chassis type of this node.  The chassis type is derived from the model number.
This is a chassis type that is compatible with Cisco's Cable Plan XML.

INPUT: None

RETURNS: str

\end{fulllineitems}

\index{get\_model() (aciphysobject.Node method)}

\begin{fulllineitems}
\phantomsection\label{aciphysobject:aciphysobject.Node.get_model}\pysiglinewithargsret{\bfcode{get\_model}}{}{}
Returns the model string of the node'

INPUT:None

RETURNS: str

\end{fulllineitems}

\index{get\_role() (aciphysobject.Node method)}

\begin{fulllineitems}
\phantomsection\label{aciphysobject:aciphysobject.Node.get_role}\pysiglinewithargsret{\bfcode{get\_role}}{}{}
retrieves the node role

\end{fulllineitems}

\index{populate\_children() (aciphysobject.Node method)}

\begin{fulllineitems}
\phantomsection\label{aciphysobject:aciphysobject.Node.populate_children}\pysiglinewithargsret{\bfcode{populate\_children}}{\emph{deep=False}}{}
will populate all of the children modules such as linecards, fantrays and powersupplies, of the node.

\end{fulllineitems}


\end{fulllineitems}

\index{Pod (class in aciphysobject)}

\begin{fulllineitems}
\phantomsection\label{aciphysobject:aciphysobject.Pod}\pysiglinewithargsret{\strong{class }\code{aciphysobject.}\bfcode{Pod}}{\emph{pod\_id}, \emph{parent=None}}{}
Bases: {\hyperref[aciphysobject:aciphysobject.BaseACIPhysObject]{\code{aciphysobject.BaseACIPhysObject}}}

Pod :  roughly equivalent to fabricPod

Initialize the basic object.  It will create the name of the pod and set the type
before calling the base class \_\_init\_\_ method.  Typically the pod\_id will be 1.
\index{get() (aciphysobject.Pod static method)}

\begin{fulllineitems}
\phantomsection\label{aciphysobject:aciphysobject.Pod.get}\pysiglinewithargsret{\strong{static }\bfcode{get}}{\emph{session}}{}
Gets all of the Pods from the APIC.  Generally there will be only one.

\end{fulllineitems}

\index{populate\_children() (aciphysobject.Pod method)}

\begin{fulllineitems}
\phantomsection\label{aciphysobject:aciphysobject.Pod.populate_children}\pysiglinewithargsret{\bfcode{populate\_children}}{\emph{deep=False}}{}
This will cause all of children of the pod to be gotten from the APIC and
populated as children of the pod.

If deep is set to True, it will populate the entire tree.

This method returns nothing.

\end{fulllineitems}


\end{fulllineitems}

\index{Powersupply (class in aciphysobject)}

\begin{fulllineitems}
\phantomsection\label{aciphysobject:aciphysobject.Powersupply}\pysiglinewithargsret{\strong{class }\code{aciphysobject.}\bfcode{Powersupply}}{\emph{pod}, \emph{node}, \emph{slot}, \emph{parent=None}}{}
Bases: {\hyperref[aciphysobject:aciphysobject.BaseACIPhysModule]{\code{aciphysobject.BaseACIPhysModule}}}

class for a power supply in a node

Initialize the basic object.  It will create the name of the powersupply and set the type
before calling the base class \_\_init\_\_ method
\index{get() (aciphysobject.Powersupply class method)}

\begin{fulllineitems}
\phantomsection\label{aciphysobject:aciphysobject.Powersupply.get}\pysiglinewithargsret{\strong{classmethod }\bfcode{get}}{\emph{session}, \emph{parent=None}}{}
Gets all of the power supplies from the APIC.  If parent is specified, it will only get power supplies that are
children of the the parent.  The power supplies will also be added as children to the parent Node.

The Powersupply object is derived mostly from the APIC `eqptPsu' class.

INPUT: session, parent of class Node

OUTPUT: list of powersupplies

\end{fulllineitems}

\index{populate\_children() (aciphysobject.Powersupply method)}

\begin{fulllineitems}
\phantomsection\label{aciphysobject:aciphysobject.Powersupply.populate_children}\pysiglinewithargsret{\bfcode{populate\_children}}{\emph{deep=False}}{}
\end{fulllineitems}


\end{fulllineitems}

\index{Supervisorcard (class in aciphysobject)}

\begin{fulllineitems}
\phantomsection\label{aciphysobject:aciphysobject.Supervisorcard}\pysiglinewithargsret{\strong{class }\code{aciphysobject.}\bfcode{Supervisorcard}}{\emph{pod}, \emph{node}, \emph{slot}, \emph{parent=None}}{}
Bases: {\hyperref[aciphysobject:aciphysobject.BaseACIPhysModule]{\code{aciphysobject.BaseACIPhysModule}}}

class representing the supervisor card of a switch

Initialize the basic object.  This should be called by the
init routines of inheriting subclasses.
\index{get() (aciphysobject.Supervisorcard class method)}

\begin{fulllineitems}
\phantomsection\label{aciphysobject:aciphysobject.Supervisorcard.get}\pysiglinewithargsret{\strong{classmethod }\bfcode{get}}{\emph{session}, \emph{parent=None}}{}
Gets all of the supervisor cards from the APIC.  If parent is specified, it will only get the supervisor card that is
a child of the the parent Node.  The supervisor will also be added as a child to the parent Node.

The Supervisorcard object is derived mostly from the APIC `eqptSupC' class.

INPUT: session, parent of class Node

OUTPUT: list of linecards

\end{fulllineitems}


\end{fulllineitems}

\index{Systemcontroller (class in aciphysobject)}

\begin{fulllineitems}
\phantomsection\label{aciphysobject:aciphysobject.Systemcontroller}\pysiglinewithargsret{\strong{class }\code{aciphysobject.}\bfcode{Systemcontroller}}{\emph{pod}, \emph{node}, \emph{slot}, \emph{parent=None}}{}
Bases: {\hyperref[aciphysobject:aciphysobject.BaseACIPhysModule]{\code{aciphysobject.BaseACIPhysModule}}}

class of the motherboard of the APIC controller node

Initialize the basic object.  It will create the name of the Systemcontroller and set the type
before calling the base class \_\_init\_\_ method.
\index{get() (aciphysobject.Systemcontroller class method)}

\begin{fulllineitems}
\phantomsection\label{aciphysobject:aciphysobject.Systemcontroller.get}\pysiglinewithargsret{\strong{classmethod }\bfcode{get}}{\emph{session}, \emph{parent=None}}{}
Gets all of the System controllers from the APIC. This information comes from
the APIC `eqptBoard' class.

If parent is specified, it will only get system controllers that are
children of the the parent.  The system controlles will also be added as children to the parent Node.

INPUT: session, parent of class Node

OUTPUT: list of Systemcontrollers

\end{fulllineitems}


\end{fulllineitems}



\section{acisession module}
\label{acisession:acisession-module}\label{acisession::doc}\label{acisession:module-acisession}\index{acisession (module)}
This module contains the Session class that controls communication
with the APIC.
\index{Session (class in acisession)}

\begin{fulllineitems}
\phantomsection\label{acisession:acisession.Session}\pysiglinewithargsret{\strong{class }\code{acisession.}\bfcode{Session}}{\emph{ipaddr}, \emph{uid}, \emph{pwd}, \emph{verify\_ssl=False}}{}
Bases: \code{object}

Session class
This class is responsible for all communication with the APIC.
\begin{quote}\begin{description}
\item[{Parameters}] \leavevmode\begin{itemize}
\item {} 
\textbf{ipaddr} -- String containing the APIC IP address in dotted        decimal notation.

\item {} 
\textbf{uid} -- String containing the username that will be used as        part of the  the APIC login credentials.

\item {} 
\textbf{pwd} -- String containing the password that will be used as        part of the  the APIC login credentials.

\item {} 
\textbf{verify\_ssl} -- Used only for SSL connections with the APIC.        Indicates whether SSL certificates must be verified.  Possible        values are True and False with the default being False.

\end{itemize}

\end{description}\end{quote}
\index{get() (acisession.Session method)}

\begin{fulllineitems}
\phantomsection\label{acisession:acisession.Session.get}\pysiglinewithargsret{\bfcode{get}}{\emph{url}}{}
Perform a REST GET call to the APIC.

\end{fulllineitems}

\index{login() (acisession.Session method)}

\begin{fulllineitems}
\phantomsection\label{acisession:acisession.Session.login}\pysiglinewithargsret{\bfcode{login}}{}{}
Initiate login to the APIC.  Opens a communication session with the        APIC using the python requests library.
\begin{quote}\begin{description}
\item[{Returns}] \leavevmode
Response class instance from the requests library.        response.ok is True if login is successful.

\end{description}\end{quote}

\end{fulllineitems}

\index{push\_to\_apic() (acisession.Session method)}

\begin{fulllineitems}
\phantomsection\label{acisession:acisession.Session.push_to_apic}\pysiglinewithargsret{\bfcode{push\_to\_apic}}{\emph{url}, \emph{data}}{}
Push the object to the APIC

\end{fulllineitems}


\end{fulllineitems}



\section{acitoolkit module}
\label{acitoolkit::doc}\label{acitoolkit:module-acitoolkit}\label{acitoolkit:acitoolkit-module}\index{acitoolkit (module)}
Main ACI Toolkit module
This is the main module that comprises the ACI Toolkit.
\index{AppProfile (class in acitoolkit)}

\begin{fulllineitems}
\phantomsection\label{acitoolkit:acitoolkit.AppProfile}\pysiglinewithargsret{\strong{class }\code{acitoolkit.}\bfcode{AppProfile}}{\emph{name}, \emph{parent}}{}
Bases: {\hyperref[acibaseobject:acibaseobject.BaseACIObject]{\code{acibaseobject.BaseACIObject}}}

The AppProfile class is used to represent the Application Profiles within
the acitoolkit object model.  In the APIC model, this class is roughly
equivalent to the fvAp class.
\begin{quote}\begin{description}
\item[{Parameters}] \leavevmode\begin{itemize}
\item {} 
\textbf{name} -- String containing the Application Profile name

\item {} 
\textbf{parent} -- An instance of Tenant class representing the Tenant        which contains this Application Profile.

\end{itemize}

\end{description}\end{quote}
\index{get() (acitoolkit.AppProfile class method)}

\begin{fulllineitems}
\phantomsection\label{acitoolkit:acitoolkit.AppProfile.get}\pysiglinewithargsret{\strong{classmethod }\bfcode{get}}{\emph{session}, \emph{tenant}}{}
Gets all of the Application Profiles from the APIC.
\begin{quote}\begin{description}
\item[{Parameters}] \leavevmode\begin{itemize}
\item {} 
\textbf{session} -- the instance of Session used for APIC communication

\item {} 
\textbf{tenant} -- the instance of Tenant used to limit the Application        Profiles retreived from the APIC

\end{itemize}

\item[{Returns}] \leavevmode
List of AppProfile objects

\end{description}\end{quote}

\end{fulllineitems}

\index{get\_json() (acitoolkit.AppProfile method)}

\begin{fulllineitems}
\phantomsection\label{acitoolkit:acitoolkit.AppProfile.get_json}\pysiglinewithargsret{\bfcode{get\_json}}{}{}
Returns json representation of the AppProfile object.
\begin{quote}\begin{description}
\item[{Returns}] \leavevmode
json dictionary of fvAp

\end{description}\end{quote}

\end{fulllineitems}


\end{fulllineitems}

\index{BaseContract (class in acitoolkit)}

\begin{fulllineitems}
\phantomsection\label{acitoolkit:acitoolkit.BaseContract}\pysiglinewithargsret{\strong{class }\code{acitoolkit.}\bfcode{BaseContract}}{\emph{contract\_name}, \emph{contract\_type='vzBrCP'}, \emph{parent=None}}{}
Bases: {\hyperref[acibaseobject:acibaseobject.BaseACIObject]{\code{acibaseobject.BaseACIObject}}}

BaseContract :  Base class for Contracts and Taboos
\index{get\_json() (acitoolkit.BaseContract method)}

\begin{fulllineitems}
\phantomsection\label{acitoolkit:acitoolkit.BaseContract.get_json}\pysiglinewithargsret{\bfcode{get\_json}}{}{}
Returns json representation of the contract

INPUT:
RETURNS: json dictionary of the contract

\end{fulllineitems}

\index{get\_scope() (acitoolkit.BaseContract method)}

\begin{fulllineitems}
\phantomsection\label{acitoolkit:acitoolkit.BaseContract.get_scope}\pysiglinewithargsret{\bfcode{get\_scope}}{}{}
Get the scope of this contract.
Valid values are `context', `global', `tenant', and
`application-profile'

\end{fulllineitems}

\index{set\_scope() (acitoolkit.BaseContract method)}

\begin{fulllineitems}
\phantomsection\label{acitoolkit:acitoolkit.BaseContract.set_scope}\pysiglinewithargsret{\bfcode{set\_scope}}{\emph{scope}}{}
Set the scope of this contract.
Valid values are `context', `global', `tenant', and
`application-profile'

\end{fulllineitems}


\end{fulllineitems}

\index{BaseInterface (class in acitoolkit)}

\begin{fulllineitems}
\phantomsection\label{acitoolkit:acitoolkit.BaseInterface}\pysiglinewithargsret{\strong{class }\code{acitoolkit.}\bfcode{BaseInterface}}{\emph{name}, \emph{parent=None}}{}
Bases: {\hyperref[acibaseobject:acibaseobject.BaseACIObject]{\code{acibaseobject.BaseACIObject}}}

Abstract class used to provide base functionality to other Interface
classes.

Initialize the basic object.  This should be called by the
init routines of inheriting subclasses.
\index{get\_port\_channel\_selector\_json() (acitoolkit.BaseInterface method)}

\begin{fulllineitems}
\phantomsection\label{acitoolkit:acitoolkit.BaseInterface.get_port_channel_selector_json}\pysiglinewithargsret{\bfcode{get\_port\_channel\_selector\_json}}{\emph{port\_name}}{}
\end{fulllineitems}

\index{get\_port\_selector\_json() (acitoolkit.BaseInterface method)}

\begin{fulllineitems}
\phantomsection\label{acitoolkit:acitoolkit.BaseInterface.get_port_selector_json}\pysiglinewithargsret{\bfcode{get\_port\_selector\_json}}{}{}
\end{fulllineitems}


\end{fulllineitems}

\index{BridgeDomain (class in acitoolkit)}

\begin{fulllineitems}
\phantomsection\label{acitoolkit:acitoolkit.BridgeDomain}\pysiglinewithargsret{\strong{class }\code{acitoolkit.}\bfcode{BridgeDomain}}{\emph{bd\_name}, \emph{parent=None}}{}
Bases: {\hyperref[acibaseobject:acibaseobject.BaseACIObject]{\code{acibaseobject.BaseACIObject}}}

BridgeDomain :  roughly equivalent to fvBD
\index{add\_context() (acitoolkit.BridgeDomain method)}

\begin{fulllineitems}
\phantomsection\label{acitoolkit:acitoolkit.BridgeDomain.add_context}\pysiglinewithargsret{\bfcode{add\_context}}{\emph{context}}{}
Set the Context for this BD

\end{fulllineitems}

\index{add\_subnet() (acitoolkit.BridgeDomain method)}

\begin{fulllineitems}
\phantomsection\label{acitoolkit:acitoolkit.BridgeDomain.add_subnet}\pysiglinewithargsret{\bfcode{add\_subnet}}{\emph{subnet}}{}
Add a subnet to this BD

\end{fulllineitems}

\index{get() (acitoolkit.BridgeDomain class method)}

\begin{fulllineitems}
\phantomsection\label{acitoolkit:acitoolkit.BridgeDomain.get}\pysiglinewithargsret{\strong{classmethod }\bfcode{get}}{\emph{session}, \emph{tenant}}{}
Gets all of the Bridge Domains from the APIC.

\end{fulllineitems}

\index{get\_context() (acitoolkit.BridgeDomain method)}

\begin{fulllineitems}
\phantomsection\label{acitoolkit:acitoolkit.BridgeDomain.get_context}\pysiglinewithargsret{\bfcode{get\_context}}{}{}
Get the Context for this BD

\end{fulllineitems}

\index{get\_json() (acitoolkit.BridgeDomain method)}

\begin{fulllineitems}
\phantomsection\label{acitoolkit:acitoolkit.BridgeDomain.get_json}\pysiglinewithargsret{\bfcode{get\_json}}{}{}
Returns json representation of the bridge domain

INPUT:
RETURNS: json dictionary of bridge domain

\end{fulllineitems}

\index{get\_subnets() (acitoolkit.BridgeDomain method)}

\begin{fulllineitems}
\phantomsection\label{acitoolkit:acitoolkit.BridgeDomain.get_subnets}\pysiglinewithargsret{\bfcode{get\_subnets}}{}{}
Get all of the subnets on this BD

\end{fulllineitems}

\index{has\_context() (acitoolkit.BridgeDomain method)}

\begin{fulllineitems}
\phantomsection\label{acitoolkit:acitoolkit.BridgeDomain.has_context}\pysiglinewithargsret{\bfcode{has\_context}}{}{}
Check if the Context has been set for this BD

\end{fulllineitems}

\index{has\_subnet() (acitoolkit.BridgeDomain method)}

\begin{fulllineitems}
\phantomsection\label{acitoolkit:acitoolkit.BridgeDomain.has_subnet}\pysiglinewithargsret{\bfcode{has\_subnet}}{\emph{subnet}}{}
Check if the BD has this particular subnet

\end{fulllineitems}

\index{remove\_context() (acitoolkit.BridgeDomain method)}

\begin{fulllineitems}
\phantomsection\label{acitoolkit:acitoolkit.BridgeDomain.remove_context}\pysiglinewithargsret{\bfcode{remove\_context}}{}{}
Remove the Context for this BD

\end{fulllineitems}

\index{remove\_subnet() (acitoolkit.BridgeDomain method)}

\begin{fulllineitems}
\phantomsection\label{acitoolkit:acitoolkit.BridgeDomain.remove_subnet}\pysiglinewithargsret{\bfcode{remove\_subnet}}{\emph{subnet}}{}
Remove a subnet from this BD

\end{fulllineitems}


\end{fulllineitems}

\index{CommonEPG (class in acitoolkit)}

\begin{fulllineitems}
\phantomsection\label{acitoolkit:acitoolkit.CommonEPG}\pysiglinewithargsret{\strong{class }\code{acitoolkit.}\bfcode{CommonEPG}}{\emph{epg\_name}, \emph{parent=None}}{}
Bases: {\hyperref[acibaseobject:acibaseobject.BaseACIObject]{\code{acibaseobject.BaseACIObject}}}

Base class for EPG and OutsideEPG.
Not meant to be instantiated directly
\index{consume() (acitoolkit.CommonEPG method)}

\begin{fulllineitems}
\phantomsection\label{acitoolkit:acitoolkit.CommonEPG.consume}\pysiglinewithargsret{\bfcode{consume}}{\emph{contract}}{}
Make this EPG consume a Contract

INPUT: Contract
RETURNS: True

\end{fulllineitems}

\index{does\_consume() (acitoolkit.CommonEPG method)}

\begin{fulllineitems}
\phantomsection\label{acitoolkit:acitoolkit.CommonEPG.does_consume}\pysiglinewithargsret{\bfcode{does\_consume}}{\emph{contract}}{}
Check if this EPG consumes a specific Contract

INPUT: Contract
RETURNS: boolean

\end{fulllineitems}

\index{does\_provide() (acitoolkit.CommonEPG method)}

\begin{fulllineitems}
\phantomsection\label{acitoolkit:acitoolkit.CommonEPG.does_provide}\pysiglinewithargsret{\bfcode{does\_provide}}{\emph{contract}}{}
Check if this EPG provides a specific Contract.
True if the EPG does provide the Contract

INPUT: Contract
RETURNS: boolean

\end{fulllineitems}

\index{dont\_consume() (acitoolkit.CommonEPG method)}

\begin{fulllineitems}
\phantomsection\label{acitoolkit:acitoolkit.CommonEPG.dont_consume}\pysiglinewithargsret{\bfcode{dont\_consume}}{\emph{contract}}{}
Make this EPG not consume a Contract.  It does not check to see
if the Contract was already consumed

INPUT: Contract
RETURNS: True

\end{fulllineitems}

\index{dont\_provide() (acitoolkit.CommonEPG method)}

\begin{fulllineitems}
\phantomsection\label{acitoolkit:acitoolkit.CommonEPG.dont_provide}\pysiglinewithargsret{\bfcode{dont\_provide}}{\emph{contract}}{}
Make this EPG not provide a Contract

INPUT: Contract
RETURNS: True

\end{fulllineitems}

\index{get() (acitoolkit.CommonEPG class method)}

\begin{fulllineitems}
\phantomsection\label{acitoolkit:acitoolkit.CommonEPG.get}\pysiglinewithargsret{\strong{classmethod }\bfcode{get}}{\emph{session}, \emph{parent}, \emph{tenant}}{}
Gets all of the EPGs from the APIC.

INPUT: session, parent, Tenant
RETURNS: List of CommonEPG

\end{fulllineitems}

\index{get\_all\_consumed() (acitoolkit.CommonEPG method)}

\begin{fulllineitems}
\phantomsection\label{acitoolkit:acitoolkit.CommonEPG.get_all_consumed}\pysiglinewithargsret{\bfcode{get\_all\_consumed}}{}{}
Get all of the Contracts consumed by this EPG

INPUT:
RETURNS: List of Contract objects

\end{fulllineitems}

\index{get\_all\_provided() (acitoolkit.CommonEPG method)}

\begin{fulllineitems}
\phantomsection\label{acitoolkit:acitoolkit.CommonEPG.get_all_provided}\pysiglinewithargsret{\bfcode{get\_all\_provided}}{}{}
Get all of the Contracts provided by this EPG

INPUT:
RETURNS: List of Contract objects

\end{fulllineitems}

\index{get\_interfaces() (acitoolkit.CommonEPG method)}

\begin{fulllineitems}
\phantomsection\label{acitoolkit:acitoolkit.CommonEPG.get_interfaces}\pysiglinewithargsret{\bfcode{get\_interfaces}}{\emph{status='attached'}}{}
Get all of the interfaces that this EPG is attached
The default is to get list of `attached' interfaces.
If `status' is set to `detached' it will return the list of
detached Interface objects

INPUT: {[}status{]} defaults to `attached'
RETURNS: List of Interface objects

\end{fulllineitems}

\index{provide() (acitoolkit.CommonEPG method)}

\begin{fulllineitems}
\phantomsection\label{acitoolkit:acitoolkit.CommonEPG.provide}\pysiglinewithargsret{\bfcode{provide}}{\emph{contract}}{}
Make this EPG provide a Contract

INPUT: Contract
RETURNS: True

\end{fulllineitems}


\end{fulllineitems}

\index{Context (class in acitoolkit)}

\begin{fulllineitems}
\phantomsection\label{acitoolkit:acitoolkit.Context}\pysiglinewithargsret{\strong{class }\code{acitoolkit.}\bfcode{Context}}{\emph{context\_name}, \emph{parent=None}}{}
Bases: {\hyperref[acibaseobject:acibaseobject.BaseACIObject]{\code{acibaseobject.BaseACIObject}}}

Context :  roughly equivalent to fvCtx
\index{get() (acitoolkit.Context class method)}

\begin{fulllineitems}
\phantomsection\label{acitoolkit:acitoolkit.Context.get}\pysiglinewithargsret{\strong{classmethod }\bfcode{get}}{\emph{session}, \emph{tenant}}{}
Gets all of the Contexts from the APIC.

\end{fulllineitems}

\index{get\_allow\_all() (acitoolkit.Context method)}

\begin{fulllineitems}
\phantomsection\label{acitoolkit:acitoolkit.Context.get_allow_all}\pysiglinewithargsret{\bfcode{get\_allow\_all}}{}{}
Get the allow\_all value.
When set, contracts will not be enforced in this context.

\end{fulllineitems}

\index{get\_json() (acitoolkit.Context method)}

\begin{fulllineitems}
\phantomsection\label{acitoolkit:acitoolkit.Context.get_json}\pysiglinewithargsret{\bfcode{get\_json}}{}{}
Returns json representation of fvCtx object

INPUT:
RETURNS: json dictionary of fvCtx object

\end{fulllineitems}

\index{set\_allow\_all() (acitoolkit.Context method)}

\begin{fulllineitems}
\phantomsection\label{acitoolkit:acitoolkit.Context.set_allow_all}\pysiglinewithargsret{\bfcode{set\_allow\_all}}{\emph{value=True}}{}
Set the allow\_all value.
When set, contracts will not be enforced in this context.

\end{fulllineitems}


\end{fulllineitems}

\index{Contract (class in acitoolkit)}

\begin{fulllineitems}
\phantomsection\label{acitoolkit:acitoolkit.Contract}\pysiglinewithargsret{\strong{class }\code{acitoolkit.}\bfcode{Contract}}{\emph{contract\_name}, \emph{parent=None}}{}
Bases: {\hyperref[acitoolkit:acitoolkit.BaseContract]{\code{acitoolkit.BaseContract}}}

Contract :  Class for Contracts
\index{get() (acitoolkit.Contract class method)}

\begin{fulllineitems}
\phantomsection\label{acitoolkit:acitoolkit.Contract.get}\pysiglinewithargsret{\strong{classmethod }\bfcode{get}}{\emph{session}, \emph{tenant}}{}
Gets all of the Contracts from the APIC for a particular tenant.

\end{fulllineitems}


\end{fulllineitems}

\index{EPG (class in acitoolkit)}

\begin{fulllineitems}
\phantomsection\label{acitoolkit:acitoolkit.EPG}\pysiglinewithargsret{\strong{class }\code{acitoolkit.}\bfcode{EPG}}{\emph{epg\_name}, \emph{parent=None}}{}
Bases: {\hyperref[acitoolkit:acitoolkit.CommonEPG]{\code{acitoolkit.CommonEPG}}}

EPG :  roughly equivalent to fvAEPg
\begin{description}
\item[{Initializes the EPG with a name and, optionally,}] \leavevmode
an AppProfile parent.
If the parent is specified and is not an AppProfile,
an error will occur.

\end{description}

INPUT: string, {[}AppProfile{]}
RETURNS:
\index{add\_bd() (acitoolkit.EPG method)}

\begin{fulllineitems}
\phantomsection\label{acitoolkit:acitoolkit.EPG.add_bd}\pysiglinewithargsret{\bfcode{add\_bd}}{\emph{bridgedomain}}{}
Add BridgeDomain to the EPG, roughly equivalent to fvRsBd

INPUT: BridgeDomain
RETURNS:

\end{fulllineitems}

\index{get\_bd() (acitoolkit.EPG method)}

\begin{fulllineitems}
\phantomsection\label{acitoolkit:acitoolkit.EPG.get_bd}\pysiglinewithargsret{\bfcode{get\_bd}}{}{}~\begin{description}
\item[{Return the assigned BridgeDomain.}] \leavevmode
There should only be one item in the returned list.

\end{description}

INPUT:
RETURNS: List of BridgeDomain objects

\end{fulllineitems}

\index{get\_json() (acitoolkit.EPG method)}

\begin{fulllineitems}
\phantomsection\label{acitoolkit:acitoolkit.EPG.get_json}\pysiglinewithargsret{\bfcode{get\_json}}{}{}
Returns json representation of the EPG

INPUT:
RETURNS: json dictionary of the EPG

\end{fulllineitems}

\index{has\_bd() (acitoolkit.EPG method)}

\begin{fulllineitems}
\phantomsection\label{acitoolkit:acitoolkit.EPG.has_bd}\pysiglinewithargsret{\bfcode{has\_bd}}{}{}
Check if a BridgeDomain has been assigned to the EPG

INPUT:
RETURNS: boolean

\end{fulllineitems}

\index{remove\_bd() (acitoolkit.EPG method)}

\begin{fulllineitems}
\phantomsection\label{acitoolkit:acitoolkit.EPG.remove_bd}\pysiglinewithargsret{\bfcode{remove\_bd}}{}{}~\begin{description}
\item[{Remove BridgeDomain from the EPG.}] \leavevmode
Note that there should only be one BridgeDomain attached
to the EPG.

\end{description}

INPUT:
RETURNS:

\end{fulllineitems}


\end{fulllineitems}

\index{Endpoint (class in acitoolkit)}

\begin{fulllineitems}
\phantomsection\label{acitoolkit:acitoolkit.Endpoint}\pysiglinewithargsret{\strong{class }\code{acitoolkit.}\bfcode{Endpoint}}{\emph{name}, \emph{parent=None}}{}
Bases: {\hyperref[acibaseobject:acibaseobject.BaseACIObject]{\code{acibaseobject.BaseACIObject}}}

Initialize the basic object.  This should be called by the
init routines of inheriting subclasses.
\index{get() (acitoolkit.Endpoint static method)}

\begin{fulllineitems}
\phantomsection\label{acitoolkit:acitoolkit.Endpoint.get}\pysiglinewithargsret{\strong{static }\bfcode{get}}{\emph{session}}{}
Gets all of the endpoints connected to the fabric from the APIC

\end{fulllineitems}


\end{fulllineitems}

\index{FilterEntry (class in acitoolkit)}

\begin{fulllineitems}
\phantomsection\label{acitoolkit:acitoolkit.FilterEntry}\pysiglinewithargsret{\strong{class }\code{acitoolkit.}\bfcode{FilterEntry}}{\emph{name}, \emph{applyToFrag}, \emph{arpOpc}, \emph{dFromPort}, \emph{dToPort}, \emph{etherT}, \emph{prot}, \emph{sFromPort}, \emph{sToPort}, \emph{tcpRules}, \emph{parent}}{}
Bases: {\hyperref[acibaseobject:acibaseobject.BaseACIObject]{\code{acibaseobject.BaseACIObject}}}

FilterEntry :  roughly equivalent to vzEntry
\index{get\_json() (acitoolkit.FilterEntry method)}

\begin{fulllineitems}
\phantomsection\label{acitoolkit:acitoolkit.FilterEntry.get_json}\pysiglinewithargsret{\bfcode{get\_json}}{}{}
Returns json representation of the FilterEntry

INPUT:
RETURNS: json dictionary of the FilterEntry

\end{fulllineitems}


\end{fulllineitems}

\index{Interface (class in acitoolkit)}

\begin{fulllineitems}
\phantomsection\label{acitoolkit:acitoolkit.Interface}\pysiglinewithargsret{\strong{class }\code{acitoolkit.}\bfcode{Interface}}{\emph{interface\_type}, \emph{pod}, \emph{node}, \emph{module}, \emph{port}, \emph{parent=None}}{}
Bases: {\hyperref[acitoolkit:acitoolkit.BaseInterface]{\code{acitoolkit.BaseInterface}}}

This class defines a physical interface.
\index{get() (acitoolkit.Interface static method)}

\begin{fulllineitems}
\phantomsection\label{acitoolkit:acitoolkit.Interface.get}\pysiglinewithargsret{\strong{static }\bfcode{get}}{\emph{session}, \emph{parent=None}}{}
Gets all of the physical interfaces from the APIC if no parent is specified.
If a parent, of type Linecard is specified, then only those interfaces on
that linecard are returned and they are also added as children to that linecard.

INPUT: session=Session, {[}parent=Linecard{]}

RETURNS: list of Interface

\end{fulllineitems}

\index{get\_json() (acitoolkit.Interface method)}

\begin{fulllineitems}
\phantomsection\label{acitoolkit:acitoolkit.Interface.get_json}\pysiglinewithargsret{\bfcode{get\_json}}{}{}
Get the json for an interface.  Returns a tuple since the json is
required to be sent in 2 posts.

\end{fulllineitems}

\index{get\_name\_for\_json() (acitoolkit.Interface method)}

\begin{fulllineitems}
\phantomsection\label{acitoolkit:acitoolkit.Interface.get_name_for_json}\pysiglinewithargsret{\bfcode{get\_name\_for\_json}}{}{}
\end{fulllineitems}

\index{get\_path() (acitoolkit.Interface method)}

\begin{fulllineitems}
\phantomsection\label{acitoolkit:acitoolkit.Interface.get_path}\pysiglinewithargsret{\bfcode{get\_path}}{}{}
Get the path of this interface used when communicating with
the APIC object model.

\end{fulllineitems}

\index{get\_serial() (acitoolkit.Interface method)}

\begin{fulllineitems}
\phantomsection\label{acitoolkit:acitoolkit.Interface.get_serial}\pysiglinewithargsret{\bfcode{get\_serial}}{}{}
\end{fulllineitems}

\index{get\_type() (acitoolkit.Interface method)}

\begin{fulllineitems}
\phantomsection\label{acitoolkit:acitoolkit.Interface.get_type}\pysiglinewithargsret{\bfcode{get\_type}}{}{}
\end{fulllineitems}

\index{get\_url() (acitoolkit.Interface method)}

\begin{fulllineitems}
\phantomsection\label{acitoolkit:acitoolkit.Interface.get_url}\pysiglinewithargsret{\bfcode{get\_url}}{}{}
\end{fulllineitems}

\index{is\_interface() (acitoolkit.Interface method)}

\begin{fulllineitems}
\phantomsection\label{acitoolkit:acitoolkit.Interface.is_interface}\pysiglinewithargsret{\bfcode{is\_interface}}{}{}
\end{fulllineitems}

\index{parse\_dn() (acitoolkit.Interface static method)}

\begin{fulllineitems}
\phantomsection\label{acitoolkit:acitoolkit.Interface.parse_dn}\pysiglinewithargsret{\strong{static }\bfcode{parse\_dn}}{\emph{dist\_name}}{}
Parses the pod, node, module, port from a
distinguished name of the interface.

\end{fulllineitems}

\index{parse\_name() (acitoolkit.Interface static method)}

\begin{fulllineitems}
\phantomsection\label{acitoolkit:acitoolkit.Interface.parse_name}\pysiglinewithargsret{\strong{static }\bfcode{parse\_name}}{\emph{name}}{}
Parses a name that is of the form:
\textless{}type\textgreater{} \textless{}pod\textgreater{}/\textless{}mod\textgreater{}/\textless{}port\textgreater{}

\end{fulllineitems}


\end{fulllineitems}

\index{L2Interface (class in acitoolkit)}

\begin{fulllineitems}
\phantomsection\label{acitoolkit:acitoolkit.L2Interface}\pysiglinewithargsret{\strong{class }\code{acitoolkit.}\bfcode{L2Interface}}{\emph{name}, \emph{encap\_type}, \emph{encap\_id}}{}
Bases: {\hyperref[acibaseobject:acibaseobject.BaseACIObject]{\code{acibaseobject.BaseACIObject}}}

The L2Interface class creates an logical L2 interface that can be        attached to a physical interface. This interface defines the L2        encapsulation i.e. VLAN, VXLAN, or NVGRE
\begin{quote}\begin{description}
\item[{Parameters}] \leavevmode\begin{itemize}
\item {} 
\textbf{name} -- String containing the L2Interface instance name

\item {} 
\textbf{encap\_type} -- String containing the encapsulation type.        Valid values are `VLAN', `VXLAN', or `NVGRE'.

\item {} 
\textbf{encap\_id} -- String containing the encapsulation specific        identifier representing the virtual L2 network (i.e. for VXLAN,        this is the numeric value of the VNID).

\end{itemize}

\end{description}\end{quote}
\index{get\_encap\_id() (acitoolkit.L2Interface method)}

\begin{fulllineitems}
\phantomsection\label{acitoolkit:acitoolkit.L2Interface.get_encap_id}\pysiglinewithargsret{\bfcode{get\_encap\_id}}{}{}
Get the encap\_id of the L2 interface.
The value is returned as a string and depends on the encap\_type
(i.e. VLAN VID, VXLAN VNID, or NVGRE VSID)
\begin{quote}\begin{description}
\item[{Returns}] \leavevmode
String containing encapsulation identifier value.

\end{description}\end{quote}

\end{fulllineitems}

\index{get\_encap\_type() (acitoolkit.L2Interface method)}

\begin{fulllineitems}
\phantomsection\label{acitoolkit:acitoolkit.L2Interface.get_encap_type}\pysiglinewithargsret{\bfcode{get\_encap\_type}}{}{}
Get the encap\_type of the L2 interface.
Valid values are `vlan', `vxlan', and `nvgre'
\begin{quote}\begin{description}
\item[{Returns}] \leavevmode
String containing encap\_type value.

\end{description}\end{quote}

\end{fulllineitems}

\index{get\_path() (acitoolkit.L2Interface method)}

\begin{fulllineitems}
\phantomsection\label{acitoolkit:acitoolkit.L2Interface.get_path}\pysiglinewithargsret{\bfcode{get\_path}}{}{}
Get the path of this interface used when communicating with        the APIC object model.
\begin{quote}\begin{description}
\item[{Returns}] \leavevmode
String containing the path \emph{TBD should this be \_get\_path} ?

\end{description}\end{quote}

\end{fulllineitems}

\index{is\_interface() (acitoolkit.L2Interface method)}

\begin{fulllineitems}
\phantomsection\label{acitoolkit:acitoolkit.L2Interface.is_interface}\pysiglinewithargsret{\bfcode{is\_interface}}{}{}
Returns whether this instance is considered an interface.
\begin{quote}\begin{description}
\item[{Returns}] \leavevmode
True

\end{description}\end{quote}

\end{fulllineitems}


\end{fulllineitems}

\index{L3Interface (class in acitoolkit)}

\begin{fulllineitems}
\phantomsection\label{acitoolkit:acitoolkit.L3Interface}\pysiglinewithargsret{\strong{class }\code{acitoolkit.}\bfcode{L3Interface}}{\emph{name}}{}
Bases: {\hyperref[acibaseobject:acibaseobject.BaseACIObject]{\code{acibaseobject.BaseACIObject}}}

Creates a L3 interface that can be attached to a L2 interface.
This interface defines the L3 address i.e. IPv4
\index{add\_context() (acitoolkit.L3Interface method)}

\begin{fulllineitems}
\phantomsection\label{acitoolkit:acitoolkit.L3Interface.add_context}\pysiglinewithargsret{\bfcode{add\_context}}{\emph{context}}{}
Add context to the EPG

\end{fulllineitems}

\index{get\_addr() (acitoolkit.L3Interface method)}

\begin{fulllineitems}
\phantomsection\label{acitoolkit:acitoolkit.L3Interface.get_addr}\pysiglinewithargsret{\bfcode{get\_addr}}{}{}
Get the L3 address assigned to this interface.
The address is set via the L3Interface.set\_addr() method

INPUT:
RETURNS: Address

\end{fulllineitems}

\index{get\_context() (acitoolkit.L3Interface method)}

\begin{fulllineitems}
\phantomsection\label{acitoolkit:acitoolkit.L3Interface.get_context}\pysiglinewithargsret{\bfcode{get\_context}}{}{}
Return the assigned context

\end{fulllineitems}

\index{get\_json() (acitoolkit.L3Interface method)}

\begin{fulllineitems}
\phantomsection\label{acitoolkit:acitoolkit.L3Interface.get_json}\pysiglinewithargsret{\bfcode{get\_json}}{}{}
Returns json representation of L3Interface

INPUT:
RETURNS: json dictionary of L3Interface

\end{fulllineitems}

\index{get\_l3if\_type() (acitoolkit.L3Interface method)}

\begin{fulllineitems}
\phantomsection\label{acitoolkit:acitoolkit.L3Interface.get_l3if_type}\pysiglinewithargsret{\bfcode{get\_l3if\_type}}{}{}
Get the l3if\_type of this L3 interface.
Valid values are `sub-interface', `l3-port', and `ext-svi'

\end{fulllineitems}

\index{has\_context() (acitoolkit.L3Interface method)}

\begin{fulllineitems}
\phantomsection\label{acitoolkit:acitoolkit.L3Interface.has_context}\pysiglinewithargsret{\bfcode{has\_context}}{}{}
Check if the context has been assigned

\end{fulllineitems}

\index{is\_interface() (acitoolkit.L3Interface method)}

\begin{fulllineitems}
\phantomsection\label{acitoolkit:acitoolkit.L3Interface.is_interface}\pysiglinewithargsret{\bfcode{is\_interface}}{}{}
Check if this is an interface object.

INPUTS:
RETURNS: True

\end{fulllineitems}

\index{remove\_context() (acitoolkit.L3Interface method)}

\begin{fulllineitems}
\phantomsection\label{acitoolkit:acitoolkit.L3Interface.remove_context}\pysiglinewithargsret{\bfcode{remove\_context}}{}{}
Remove context from the EPG

\end{fulllineitems}

\index{set\_addr() (acitoolkit.L3Interface method)}

\begin{fulllineitems}
\phantomsection\label{acitoolkit:acitoolkit.L3Interface.set_addr}\pysiglinewithargsret{\bfcode{set\_addr}}{\emph{addr}}{}
Set the L3 address assigned to this interface

INPUT: L3 address
RETURNS:

\end{fulllineitems}

\index{set\_l3if\_type() (acitoolkit.L3Interface method)}

\begin{fulllineitems}
\phantomsection\label{acitoolkit:acitoolkit.L3Interface.set_l3if_type}\pysiglinewithargsret{\bfcode{set\_l3if\_type}}{\emph{l3if\_type}}{}
Set the l3if\_type of this L3 interface.
Valid values are `sub-interface', `l3-port', and `ext-svi'

\end{fulllineitems}


\end{fulllineitems}

\index{NetworkPool (class in acitoolkit)}

\begin{fulllineitems}
\phantomsection\label{acitoolkit:acitoolkit.NetworkPool}\pysiglinewithargsret{\strong{class }\code{acitoolkit.}\bfcode{NetworkPool}}{\emph{name}, \emph{encap\_type}, \emph{start\_id}, \emph{end\_id}, \emph{mode}}{}
Bases: {\hyperref[acibaseobject:acibaseobject.BaseACIObject]{\code{acibaseobject.BaseACIObject}}}

This class defines a pool of network ids
\index{get\_json() (acitoolkit.NetworkPool method)}

\begin{fulllineitems}
\phantomsection\label{acitoolkit:acitoolkit.NetworkPool.get_json}\pysiglinewithargsret{\bfcode{get\_json}}{}{}
\end{fulllineitems}


\end{fulllineitems}

\index{OSPFInterface (class in acitoolkit)}

\begin{fulllineitems}
\phantomsection\label{acitoolkit:acitoolkit.OSPFInterface}\pysiglinewithargsret{\strong{class }\code{acitoolkit.}\bfcode{OSPFInterface}}{\emph{name}, \emph{area\_id=None}}{}
Bases: {\hyperref[acibaseobject:acibaseobject.BaseACIObject]{\code{acibaseobject.BaseACIObject}}}

Creates an OSPF router interface that can be attached to a L3 interface.
This interface defines the OSPF area, authentication, etc.
\index{get\_json() (acitoolkit.OSPFInterface method)}

\begin{fulllineitems}
\phantomsection\label{acitoolkit:acitoolkit.OSPFInterface.get_json}\pysiglinewithargsret{\bfcode{get\_json}}{}{}
Returns json representation of OSPFInterface

INPUT:
RETURNS: json dictionary of OSPFInterface

\end{fulllineitems}

\index{is\_interface() (acitoolkit.OSPFInterface method)}

\begin{fulllineitems}
\phantomsection\label{acitoolkit:acitoolkit.OSPFInterface.is_interface}\pysiglinewithargsret{\bfcode{is\_interface}}{}{}
\end{fulllineitems}

\index{is\_ospf() (acitoolkit.OSPFInterface static method)}

\begin{fulllineitems}
\phantomsection\label{acitoolkit:acitoolkit.OSPFInterface.is_ospf}\pysiglinewithargsret{\strong{static }\bfcode{is\_ospf}}{}{}
Returns True if this interface is an OSPF interface

\end{fulllineitems}


\end{fulllineitems}

\index{OSPFRouter (class in acitoolkit)}

\begin{fulllineitems}
\phantomsection\label{acitoolkit:acitoolkit.OSPFRouter}\pysiglinewithargsret{\strong{class }\code{acitoolkit.}\bfcode{OSPFRouter}}{\emph{name}}{}
Bases: {\hyperref[acibaseobject:acibaseobject.BaseACIObject]{\code{acibaseobject.BaseACIObject}}}

Represents the global settings of the OSPF Router

\end{fulllineitems}

\index{OutsideEPG (class in acitoolkit)}

\begin{fulllineitems}
\phantomsection\label{acitoolkit:acitoolkit.OutsideEPG}\pysiglinewithargsret{\strong{class }\code{acitoolkit.}\bfcode{OutsideEPG}}{\emph{epg\_name}, \emph{parent=None}}{}
Bases: {\hyperref[acitoolkit:acitoolkit.CommonEPG]{\code{acitoolkit.CommonEPG}}}

Represents the EPG for external connectivity
\index{get\_json() (acitoolkit.OutsideEPG method)}

\begin{fulllineitems}
\phantomsection\label{acitoolkit:acitoolkit.OutsideEPG.get_json}\pysiglinewithargsret{\bfcode{get\_json}}{}{}
Returns json representation of OutsideEPG

INPUT:
RETURNS: json dictionary of OutsideEPG

\end{fulllineitems}


\end{fulllineitems}

\index{PortChannel (class in acitoolkit)}

\begin{fulllineitems}
\phantomsection\label{acitoolkit:acitoolkit.PortChannel}\pysiglinewithargsret{\strong{class }\code{acitoolkit.}\bfcode{PortChannel}}{\emph{name}}{}
Bases: {\hyperref[acitoolkit:acitoolkit.BaseInterface]{\code{acitoolkit.BaseInterface}}}

This class defines a port channel interface.
\index{attach() (acitoolkit.PortChannel method)}

\begin{fulllineitems}
\phantomsection\label{acitoolkit:acitoolkit.PortChannel.attach}\pysiglinewithargsret{\bfcode{attach}}{\emph{interface}}{}
Attach an interface to this PortChannel

\end{fulllineitems}

\index{detach() (acitoolkit.PortChannel method)}

\begin{fulllineitems}
\phantomsection\label{acitoolkit:acitoolkit.PortChannel.detach}\pysiglinewithargsret{\bfcode{detach}}{\emph{interface}}{}
Detach an interface from this PortChannel

\end{fulllineitems}

\index{get() (acitoolkit.PortChannel static method)}

\begin{fulllineitems}
\phantomsection\label{acitoolkit:acitoolkit.PortChannel.get}\pysiglinewithargsret{\strong{static }\bfcode{get}}{\emph{session}}{}
Gets all of the port channel interfaces from the APIC

\end{fulllineitems}

\index{get\_json() (acitoolkit.PortChannel method)}

\begin{fulllineitems}
\phantomsection\label{acitoolkit:acitoolkit.PortChannel.get_json}\pysiglinewithargsret{\bfcode{get\_json}}{}{}
Returns json representation of the PortChannel

INPUT:
RETURNS: json dictionary of the PortChannel

\end{fulllineitems}

\index{get\_path() (acitoolkit.PortChannel method)}

\begin{fulllineitems}
\phantomsection\label{acitoolkit:acitoolkit.PortChannel.get_path}\pysiglinewithargsret{\bfcode{get\_path}}{}{}
Get the path of this interface used when communicating with
the APIC object model.

\end{fulllineitems}

\index{is\_interface() (acitoolkit.PortChannel method)}

\begin{fulllineitems}
\phantomsection\label{acitoolkit:acitoolkit.PortChannel.is_interface}\pysiglinewithargsret{\bfcode{is\_interface}}{}{}
Returns True since a PortChannel is an interface

\end{fulllineitems}

\index{is\_vpc() (acitoolkit.PortChannel method)}

\begin{fulllineitems}
\phantomsection\label{acitoolkit:acitoolkit.PortChannel.is_vpc}\pysiglinewithargsret{\bfcode{is\_vpc}}{}{}
Returns True if the PortChannel is a VPC

\end{fulllineitems}


\end{fulllineitems}

\index{Search (class in acitoolkit)}

\begin{fulllineitems}
\phantomsection\label{acitoolkit:acitoolkit.Search}\pysigline{\strong{class }\code{acitoolkit.}\bfcode{Search}}
Bases: {\hyperref[acibaseobject:acibaseobject.BaseACIObject]{\code{acibaseobject.BaseACIObject}}}

This is an empty class used to create a search object for use with the ``find'' method.

Attaching attributes to this class and then invoking find will return all objects with matching attributes
in the object hierarchy at and below where the find is invoked.

\end{fulllineitems}

\index{Subnet (class in acitoolkit)}

\begin{fulllineitems}
\phantomsection\label{acitoolkit:acitoolkit.Subnet}\pysiglinewithargsret{\strong{class }\code{acitoolkit.}\bfcode{Subnet}}{\emph{subnet\_name}, \emph{parent=None}}{}
Bases: {\hyperref[acibaseobject:acibaseobject.BaseACIObject]{\code{acibaseobject.BaseACIObject}}}

Subnet :  roughly equivalent to fvSubnet
\index{get() (acitoolkit.Subnet class method)}

\begin{fulllineitems}
\phantomsection\label{acitoolkit:acitoolkit.Subnet.get}\pysiglinewithargsret{\strong{classmethod }\bfcode{get}}{\emph{session}, \emph{bridgedomain}, \emph{tenant}}{}
Gets all of the Subnets from the APIC for a particular tenant and
bridgedomain

\end{fulllineitems}

\index{get\_addr() (acitoolkit.Subnet method)}

\begin{fulllineitems}
\phantomsection\label{acitoolkit:acitoolkit.Subnet.get_addr}\pysiglinewithargsret{\bfcode{get\_addr}}{}{}
Get the subnet address

INPUT:
RETURNS: The subnet address as a string in the form
\begin{quote}

of \textless{}ipaddr\textgreater{}/\textless{}mask\textgreater{}
\end{quote}

\end{fulllineitems}

\index{get\_json() (acitoolkit.Subnet method)}

\begin{fulllineitems}
\phantomsection\label{acitoolkit:acitoolkit.Subnet.get_json}\pysiglinewithargsret{\bfcode{get\_json}}{}{}
Returns json representation of the subnet

INPUT:
RETURNS: json dictionary of subnet

\end{fulllineitems}

\index{set\_addr() (acitoolkit.Subnet method)}

\begin{fulllineitems}
\phantomsection\label{acitoolkit:acitoolkit.Subnet.set_addr}\pysiglinewithargsret{\bfcode{set\_addr}}{\emph{addr}}{}
Set the subnet address
\begin{description}
\item[{INPUT: addr: The subnet address as a string in the form}] \leavevmode
of \textless{}ipaddr\textgreater{}/\textless{}mask\textgreater{}

\end{description}

\end{fulllineitems}


\end{fulllineitems}

\index{Taboo (class in acitoolkit)}

\begin{fulllineitems}
\phantomsection\label{acitoolkit:acitoolkit.Taboo}\pysiglinewithargsret{\strong{class }\code{acitoolkit.}\bfcode{Taboo}}{\emph{contract\_name}, \emph{parent=None}}{}
Bases: {\hyperref[acitoolkit:acitoolkit.BaseContract]{\code{acitoolkit.BaseContract}}}

Taboo :  Class for Taboos

\end{fulllineitems}

\index{Tenant (class in acitoolkit)}

\begin{fulllineitems}
\phantomsection\label{acitoolkit:acitoolkit.Tenant}\pysiglinewithargsret{\strong{class }\code{acitoolkit.}\bfcode{Tenant}}{\emph{name}, \emph{parent=None}}{}
Bases: {\hyperref[acibaseobject:acibaseobject.BaseACIObject]{\code{acibaseobject.BaseACIObject}}}

The Tenant class is used to represent the tenants within the acitoolkit
object model.  In the APIC model, this class is roughly equivalent to
the fvTenant class.

Initialize the basic object.  This should be called by the
init routines of inheriting subclasses.
\index{exists() (acitoolkit.Tenant class method)}

\begin{fulllineitems}
\phantomsection\label{acitoolkit:acitoolkit.Tenant.exists}\pysiglinewithargsret{\strong{classmethod }\bfcode{exists}}{\emph{session}, \emph{tenant}}{}
Check if a tenant exists on the APIC.
\begin{quote}\begin{description}
\item[{Parameters}] \leavevmode\begin{itemize}
\item {} 
\textbf{session} -- the instance of Session used for APIC communication

\item {} 
\textbf{tenant} -- the instance of Tenant to check if exists on the APIC

\end{itemize}

\item[{Returns}] \leavevmode
True or False

\end{description}\end{quote}

\end{fulllineitems}

\index{get() (acitoolkit.Tenant class method)}

\begin{fulllineitems}
\phantomsection\label{acitoolkit:acitoolkit.Tenant.get}\pysiglinewithargsret{\strong{classmethod }\bfcode{get}}{\emph{session}}{}
Gets all of the tenants from the APIC.
\begin{quote}\begin{description}
\item[{Parameters}] \leavevmode
\textbf{session} -- the instance of Session used for APIC communication

\item[{Returns}] \leavevmode
a list of Tenant objects

\end{description}\end{quote}

\end{fulllineitems}

\index{get\_json() (acitoolkit.Tenant method)}

\begin{fulllineitems}
\phantomsection\label{acitoolkit:acitoolkit.Tenant.get_json}\pysiglinewithargsret{\bfcode{get\_json}}{}{}
Returns json representation of the fvTenant object
\begin{quote}\begin{description}
\item[{Returns}] \leavevmode
A json dictionary of fvTenant

\end{description}\end{quote}

\end{fulllineitems}

\index{get\_url() (acitoolkit.Tenant static method)}

\begin{fulllineitems}
\phantomsection\label{acitoolkit:acitoolkit.Tenant.get_url}\pysiglinewithargsret{\strong{static }\bfcode{get\_url}}{\emph{fmt='json'}}{}
Get the URL used to push the configuration to the APIC
if no format parameter is specified, the format will be `json'
otherwise it will return `/api/mo/uni.' with the format string
appended.
\begin{quote}\begin{description}
\item[{Parameters}] \leavevmode
\textbf{fmt} -- optional format string, default is `json'

\item[{Returns}] \leavevmode
URL string

\end{description}\end{quote}

\end{fulllineitems}


\end{fulllineitems}

\index{VMM (class in acitoolkit)}

\begin{fulllineitems}
\phantomsection\label{acitoolkit:acitoolkit.VMM}\pysiglinewithargsret{\strong{class }\code{acitoolkit.}\bfcode{VMM}}{\emph{name}, \emph{ipaddr}, \emph{credentials}, \emph{vswitch\_info}, \emph{network\_pool}}{}
Bases: {\hyperref[acibaseobject:acibaseobject.BaseACIObject]{\code{acibaseobject.BaseACIObject}}}

This class defines an instance of connectivity to a
Virtual Machine Manager (such as VMware vCenter)
\index{get\_json() (acitoolkit.VMM method)}

\begin{fulllineitems}
\phantomsection\label{acitoolkit:acitoolkit.VMM.get_json}\pysiglinewithargsret{\bfcode{get\_json}}{}{}
\end{fulllineitems}

\index{get\_path() (acitoolkit.VMM method)}

\begin{fulllineitems}
\phantomsection\label{acitoolkit:acitoolkit.VMM.get_path}\pysiglinewithargsret{\bfcode{get\_path}}{}{}
\end{fulllineitems}


\end{fulllineitems}

\index{VMMCredentials (class in acitoolkit)}

\begin{fulllineitems}
\phantomsection\label{acitoolkit:acitoolkit.VMMCredentials}\pysiglinewithargsret{\strong{class }\code{acitoolkit.}\bfcode{VMMCredentials}}{\emph{name}, \emph{uid}, \emph{pwd}}{}
Bases: {\hyperref[acibaseobject:acibaseobject.BaseACIObject]{\code{acibaseobject.BaseACIObject}}}

This class defines the credentials used to login to a Virtual
Machine Manager
\index{get\_json() (acitoolkit.VMMCredentials method)}

\begin{fulllineitems}
\phantomsection\label{acitoolkit:acitoolkit.VMMCredentials.get_json}\pysiglinewithargsret{\bfcode{get\_json}}{}{}
\end{fulllineitems}


\end{fulllineitems}

\index{VMMvSwitchInfo (class in acitoolkit)}

\begin{fulllineitems}
\phantomsection\label{acitoolkit:acitoolkit.VMMvSwitchInfo}\pysiglinewithargsret{\strong{class }\code{acitoolkit.}\bfcode{VMMvSwitchInfo}}{\emph{vendor}, \emph{container\_name}, \emph{vswitch\_name}}{}
Bases: \code{object}

This class contains the information necessary for creating the
vSwitch on the Virtual Machine Manager

\end{fulllineitems}



\chapter{Indices and tables}
\label{index:indices-and-tables}\begin{itemize}
\item {} 
\emph{genindex}

\item {} 
\emph{modindex}

\item {} 
\emph{search}

\end{itemize}


\renewcommand{\indexname}{Python Module Index}
\begin{theindex}
\def\bigletter#1{{\Large\sffamily#1}\nopagebreak\vspace{1mm}}
\bigletter{a}
\item {\texttt{acibaseobject}}, \pageref{acibaseobject:module-acibaseobject}
\item {\texttt{aciphysobject}}, \pageref{aciphysobject:module-aciphysobject}
\item {\texttt{acisession}}, \pageref{acisession:module-acisession}
\item {\texttt{acitoolkit}}, \pageref{acitoolkit:module-acitoolkit}
\end{theindex}

\renewcommand{\indexname}{Index}
\printindex
\end{document}
